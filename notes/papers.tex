\section{Paper shorts}

\subsection{OMR}
\begin{itemize}
    \item Bellini 2007 \citep{Bellini2007} Overview of state-of-the-art, with a call-to-action for more research. Attempts to provide more standard metrics for evaulation of OMR systems.
    \item Bainbridge 2001 \citep{Bainbridge2001} Describes properties of music that make OMR a difficult task, review of OMR research pre-2000. Overview of sub-tasks for OMR pipeline (check if \citep{Rebelo2012} corresponds with this), with common solutions.
    \item Rebelo 2012\citep{Rebelo2012} A more detailed outline of the different subtasks that are involved in OMR, with a large set of available research linked to those subtasks. \citep{}
    \item Byrd 2015 \citep{Byrd2015} Set of definitions for complexity of music notation, set of performance metrics for page quality and score complexity, provides a small corpus of music as a baseline for a proper OMR testbed. 
    \item Raphael 2011 \citep{Raphael2011} Presents a first version of an OMR system that 1) first segments a page into systems, system measures and individual measures, and 2) uses the measures as a basic unit for recognition. Recognition is divided into rigid symbols (clefs, rests, etc) and composite symbols (notes with stems, chords, beamed groups, etc). See also \citep{Chen2016}
   \item Calvo-Zaragoza 2019 \citep{CalvoZaragoza2019} A paper that aims to be a primer in OMR. The authors argue that the OMR community is hurting from a lack of framework in which to work, missing clear task definitions and evaluation methods. It provides a taxonomy of OMR, and a taxonomy of OMR applications.
\end{itemize}

\subsection{Difficulties of OMR}
\begin{itemize}
    \item Bainbridge 2001 \citep{Bainbridge2001} Text has an ``easy'' vertical dimension, where notes in music can be translated arbitrarily. The same musical event can have different notations because of horizontal/vertical stretching. Many shapes in music are similar, in contrast to textual components. These transformations can all occur simultaneously. Furthermore the layout of musical features can influence other features in the 2-d space on a page, even far apart. Music notation can be ambiguous. Usually an experienced musician can understand the correct meaning, since other meanings would be non-sensical, computers however can not. Musical features are superimposed on staffs intentionally, since those convey pitch, and slurs, ties, etc. can sometimes cross barlines. Finally, music notation contains text, such as bar numbers, dynamics, tempo indicators, lyrics, etc. This makes OMR a superset of OCR.
    \item Bellini 2007 \citep{Bellini2007} There are open issues, since the meaning of ``Music Recognition'' changes based on the application. Comparisons of OMR results is difficult, since there is currently no standard database, and no standard terminology.
    \item Homenda 2006 \citep{Homenda2006} Scanned music notation that is to be recognized can often be blurry, noisy and fragmented. Size and shape of different music symbols can be highly irregular, groups of symbols can be very sparse or tightly packed. There is no universal pattern recognition method that could effectively be applied because of this.
    \item Fujinaga 1996 \citep{Fujinaga1996} Unlike text, music contains many symbols that vary in size and orientation. The musical alphabet is not entirely fixed, and new symbols are being added. Mentions differences in number of staff lines (1 to many). Recommends the thinnest objects (usually stems) should have a thickness of 3 to 5 pixels.
    \item Byrd 2015 \citep{Byrd2015} Music is two-dimensional, can contain several interleaved strands, each voice containing notest that have both pitch and duration. There is a high context dependency in musical symbols, both in pitch (e.g. accidentals) and time (e.g. left out repetitive symbols, such as triplet indications). Goes as far as to argue that CWMN is more complex than mathematical notation.
\end{itemize}

\subsection{OMR pipeline}
\begin{itemize}
    \item Bainbridge 2001 \citep{Bainbridge2001} 
    \item Rebelo 2012 \citep{Rebelo2012} 
\end{itemize}

\subsection{Full-pipeline or end-to-end OMR}
\begin{itemize}
    \item Calvo-Zaragoza 2017 \citep{CalvoZaragoza2017} First attempt at end-to-end OMR. Using a convolutional recurrent nn with synthetically obtained monophonic scores with limited length and symbols. Classifies concurrent symbols, but lacks polyphonic and cross-symbol spanning elements, such as slurs, ties, text, etc.
    \item Calvo-Zaragoza 2018 \citep{CalvoZaragoza2018} Builds upon previous paper, but now with a newly created dataset: PRiMuS. Results are quite good, but still limited to monophonic music, and there are some dubious claims in the conclusion: In previous paper the review of Rebelo et al. \citep{Rebelo2012} showed "results have not been very promising so far", however the task in this paper "can be solved succesfully using the considered ... approach". Also, claiming "performance comparable to that of commercial systems" after a "qualitative comparative study of selected examples" seems quite a stretch.
    \item Wel 2017 \citep{Wel2017} Convolutional and recurrent nn that translates a line of sheet music to (pitch, duration) pairs. All data is monophonic, and the resulting format discards all markup, such as beamed note groups, stem direction etc.
\end{itemize}

\subsection{Human-aided OMR}
\begin{itemize}
    \item Petros 2020 \citep{Samiotis2020} Designing micro-tasks in such a way that crowd sourcing can be used on a highly varied crowd. User interaction is through Amazon Mechanical Turk, tasks are 1 to 3 measures long, and differ between providing the score, the audio, or both. Users were asked to compare two versions of the and indicate whether they are the same. \citep{Burghardt2017, Bellini2007, Chen2016, CalvoZaragoza2019}
    \item Burghardt 2017 \citep{Burghardt2017} The Allegro system is a framework in which users can manually transcribe music scores. It was used to transcribe a collection of handwritten folk music that contains single melodic lines only. Includes references to crowd sourcing and OCR correction papers. \citep{Bainbridge2001, Bellini2007, Raphael2011, Rebelo2012}
    \item Chen 2016 \citep{Chen2016} Human directed OMR takes an existing recognition engine and brings human input into the loop. In three stages, staff recognition, system identification and measure contents recognition, users are employed to correct errors made by the system by drawing fundamentels such as stems and notes, and labelling them manually.
    \item Chen 2016(b) \citep{Chen2016b} Human interactive OMR system called Ceres that also employs human input, but works on a symbol-level instead of a measure-level. Users can add contraints to the system by identifying a pixel as a classified object, and the system re-recognizes the contents of a page within these constraints.
\end{itemize}

\subsection{OMR techniques}
\begin{itemize}
    \item Otsu 1979 \citep{Otsu1979} binarization technique

    \item Bainbridge 2001 \citep{Bainbridge2001} Staff line detection through horizontal projections, possibly only on start and end of staff line to avoid noise through music symbols.

    \item Staff line height and staff space height determined through run-length encoding \citep{Rebelo2012}

    \item Dalitz 2008 \citep{Dalitz2008} Comparing several staff removal algorithms, which are implemented in the Gamera toolkit.

    \item Vigliensoni 2013 \citep{Vigliensoni2013} Optical measure recognition technique described. Staff detection is taken from \citep{} and barline detection is developed by the authors.
\end{itemize}

\subsection{Clustering}
\begin{itemize}
    \item Senin 2008 \citep{Senin2008} Dynamic Time Warping overview, basic algorithm, customizations and optimizations with some examples.
    \item Niennattrakul 2006 \citep{Niennattrakul2006} Clustering multimedia data with time series. Comparing Euclidean and Dynamic Time Warping distances in a k-mediods clustering algorith.
    \item Niennattrakul 2007 \citep{Niennattrakul2007} Same topic as previous paper, but now showing why k-means is inferior to k-mediods clustering when using DTW measure as distance metric.
\end{itemize}


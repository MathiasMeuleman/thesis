\chapter{Related work}

\section{OMR}
\begin{itemize}
    \item OMR is an extension of OCR, but with many more complex problems. The 2 dimensional nature of OMR, in contrast to the 1 dimensional nature of OCR, can cause troubles in accurately transcribing music \citeneeded. Besides this, there is a vast collection of symbols possible in sheet music, with a very uneven occurrence distribution, resulting in a large number of false positives for the symbols that occur infrequently (Chen, Stolterman, 2016). There is also a large portion of written music that adheres strongly to a fixed ruleset, advocating for a rule-based recognition system, which is offset against the small but long tail of exceptions on these rules, making any rule-based recognition system inaccurate in too many cases.
    \item Throughout the last two decades a generally accepted OMR pipeline has been devised, first started by Bellini et al. (2004??) and later refined by Rebelo et al. (2012). Various stages of this pipeline remain unsolved.
    \item To find solution for these open issues, two trends can be seen over the last decade or so: deep learning and human-aided recognition. (include lots of sources for both pls)
    \item In deep learning, promising results are shown for certain stagese of the pipeline (Gallego, Calvo-Zaragoza, 2017), (Pacha, Calvo-Zaragoza, Hajic, 2019) as well as for a full-pipeline end-to-end learning approach (Calvo-Zaragoza, Valer-Mas, 2017), (Wel, Ullrich, 2017). The main difficulty with the deep learning approach is the cost of collecting data. There is only a small collection of datasets available, getting more of these is expensive.
    \item Meanwhile human-aided solutions have also started to get some traction. Examples are the Allegro system (Burghardt, Spanner, (2017), human-in-the-loop systems on measure level (Chen, Stolterman, 2016) and on symbol level (Chen, Raphael, 2016)
\end{itemize}

\section{Crowd computing}

\chapter{Related work}\label{chap:related-work}

\section{OMR}\label{sec:related-work-OMR}
\begin{itemize}
    \item OMR is an extension of OCR, but with many more complex problems. The 2 dimensional nature of OMR, in contrast to the 1 dimensional nature of OCR, can cause troubles in accurately transcribing music \citeneeded. Besides this, there is a vast collection of symbols possible in sheet music, with a very uneven occurrence distribution, resulting in a large number of false positives for the symbols that occur infrequently (Chen, Stolterman, 2016). There is also a large portion of written music that adheres strongly to a fixed ruleset, advocating for a rule-based recognition system, which is offset against the small but long tail of exceptions on these rules, making any rule-based recognition system inaccurate in too many cases.
    \item Throughout the last two decades a generally accepted OMR pipeline has been devised, first started by Bellini et al. (2004??) and later refined by Rebelo et al. (2012). Various stages of this pipeline remain unsolved.
    \item To find solution for these open issues, two trends can be seen over the last decade or so: deep learning and human-aided recognition. (include lots of sources for both pls)
    \item In deep learning, promising results are shown for certain stagese of the pipeline (Gallego, Calvo-Zaragoza, 2017), (Pacha, Calvo-Zaragoza, Hajic, 2019) as well as for a full-pipeline end-to-end learning approach (Calvo-Zaragoza, Valer-Mas, 2017), (Wel, Ullrich, 2017). The main difficulty with the deep learning approach is the cost of collecting data. There is only a small collection of datasets available, getting more of these is expensive.
    \item Meanwhile human-aided solutions have also started to get some traction. Examples are the Allegro system (Burghardt, Spanner, (2017), human-in-the-loop systems on measure level (Chen, Stolterman, 2016) and on symbol level (Chen, Raphael, 2016)
\end{itemize}

\section{Crowd computing}\label{sec:related-work-crowd-computing}

\section{Measure detectors}
The segmenter currently embedded in the OMR pipeline is the one taken from the work of Waloschek, Hadjakos and Pacha [1]. Their approach consisted of manually annotating measures on pages of orchestral scores, defining a distance metric between annotated measures and training a CNN to detect measures in new input data. There are a few shortcomings to this approach. First of all, this model is far from perfect and makes too many mistakes to be used in a reasonable manner in this OMR pipeline. Even on pages that contain high quality scans and straight barlines, the detection is not perfect and therefore requires post-processing, see Figure 1 for two examples. Second, the measures that are detected are restricted to separation over time only. This means that when applying this to music scores with multiple instruments or voices -which is common in orchestral music, choir music, or piano scores- the segmenter will only segment horizontally, but leaves the different voicings grouped together in the same block. Besides this, the model is fixed and cannot be easily improved upon. Retraining the model to overcome the mistakes it currently makes would require manual annotation of these pages which is a very costly process. Finally, this model is relatively slow compared to other approaches. \issue{Insert mentions of \citep{Zalkow2019}}

Figure 1: Two examples of errors of the CNN method. In the first example we see that a large part of the entire page is classified as a single measure, in the second examples we see that smaller subsections of measures are detected as measures.

\section{Existing datasets}\label{sec:related-work-existing-datasets}
When working with written music score data, there are a few datasets already available in the OMR field. Examples are the CVC-MUSCIMA \citep{Fornes2012} and its derivative, the MUSCIMA++ \citep{Hajic2017} for handwritten data, mainly aimed at staff line removal and symbol classification, and the MeasureBoundingBoxAnnotations dataset \citep{Zalkow2019}. Both the MUSCIMA++ and version 2 of the MeasureBoundingBoxAnnotations datasets have annotations for bounding boxes of individual measures.
